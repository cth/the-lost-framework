% This LaTeX document was generated using the LaTeX backend of PlDoc,
% The SWI-Prolog documentation system



\section{chmm.pl -- Constrained Hidden Markov Models}

\label{sec:chmm}

\begin{tags}
    \tag{author}
: Christian Theil Have
\end{tags}

This module is an implementation of Constrained Hidden Markov Models in PRISM. The implementation includes
a few well-known global constraints which may be used with the model. The implementation is described in detail in
ICLP 2010 paper:

Henning Christiansen, Christian Theil Have, Ole Torp Lassen and Matthieu Petit
"Inference with Constrained Hidden Markov Models in PRISM".

\textit{Abstract: A Hidden Markov Model (HMM) is a common statistical model which is widely used for analysis
of biological sequence data and other sequential phenomena. In the present paper we show how HMMs can
be extended with side-constraints and present constraint solving techniques for efficient inference.
Defining HMMs with side-constraints in Constraint Logic Programming have advantages in terms of more compact
expression and pruning opportunities during inference. We present a PRISM-based framework for extending HMMs
with side-constraints and show how well-known constraints such as cardinality and all_different are integrated.
We experimentally validate our approach on the biologically motivated problem of global pairwise alignment.}\vspace{0.7cm}

\begin{description}
    \predicate{init_store}{0}{}
Initialization of constraints added on the model

    \predicate{forward_store}{1}{?S}
The predicate get the current store or remove it

    \predicate{get_store}{1}{S}
Get the current store

    \predicate[private]{init_constraint_stores}{2}{+Constraints, -Store}
Recursive predicate that gets every initial \arg{Store} associated
with each constraint of \arg{Constraints}
\end{description}

