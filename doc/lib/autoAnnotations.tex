% This LaTeX document was generated using the LaTeX backend of PlDoc,
% The SWI-Prolog documentation system



\section{autoAnnotations.pl -- autoAnnotations}

\label{sec:autoAnnotations}

\begin{tags}
    \tag{author}
: Henning Christiansen
\end{tags}

This file defines a preprocessor for PRISM programs that include
annotations, which are redundant arguments intended to present
abstract descriptions from the data being modelled.

Such annotations can be extracted from the proof trees generated
by PRISM's viterbi predicates, but this is quite tedious to program

Such annotations in a model tend to make PRISM's viterbi calculations run very slow
(in many cases, prohibitively slow).
On the other hand, annotations are

\begin{enumerate}
    \item essential for doing supervised learning
    \item convenient when presenting predictions (most probably analyses) for the user.
The autoAnnotations snystem takes a PRISM program that includes annotations;
the user must indicate which arguments that are annotations by a special syntax
illustrated in the supplied sample file.
It can produce automatically, the following programs
    \item a version of the PRISM program without annotations

\begin{shortlist}
    \item useful for viterbi analyses
\end{shortlist}

    \item an executable version of the PRISM program with annotations

\begin{itemize}
    \item useful for initial testing of the model, for sampling and for supervised learning
Probabilities found by learning with program (2) can be used with program (1)
\end{itemize}

    \item a translator from the proof trees produced by program (1) under viterbi prediction
The main advantage of using autoAnnotations is that you need only maintain
a single program containing the 'logic' of your model.

Version 2.1 for PRISM 1.12 and higher. January 2009
Changes:

\begin{itemize}
    \item PRISM has a viterbit that produces a proper tree (so the code gets cleaner).
    \item PRISMs target declaration is obsolete - which is fine as previous version of
autoAnnotations did not treat it correctly.
\end{itemize}

Written by Henning Christiansen, henning@ruc.dk, (c) 2008
Beta version, December 2008; testing still very limited
Beta was tested extensively by students - no bugs were found

Version 2 is developed Jan 2009 - based on new facilities of PRISM 1.12

Bug fixed 30 mar 2009 \Sifthen{} v 2.1
\end{enumerate}

\begin{itemize}
    \item nb: fix is based on undocumented prism predicate '\$is_prob_pred'(Pname,Parity)
    \item Fixed to use \$pd_is_prob_pred instead of deprecated \$is_prob_pred (works with PRISM ver. 2.x)
STILL NEEDS TO BE TRIMMED FOR OPTIMAL STORAGE UTILIZATION
\end{itemize}

\vspace{0.7cm}

\begin{description}
    \predicate[det]{prismAnnot}{1}{+File}
Should be called to load a PRISM file. This will transform the progrram and subsequent
calls to \predref{viterbiAnnot}{2} will work efficiently using the transformed program.
Note that this some intermediate Prolog files files.

    \predicate[det]{prismAnnot}{2}{+File, +Mode}
\arg{Mode} is \verb$direct$ or \verb$separate$
For learning you need to work with the 'direct' program, and then
put probabilities into a file - and then load them in when you have compiler
the separate program.
Using both \verb$prismAnnot($\arg{File}\verb$,separate)$ and \verb$prismAnnot($\arg{File}\verb$,direct)$ at the same
time is not possible; only the most recent one counts.

    \predicate[det]{viterbiAnnot}{2}{+Call, -P}
viterbiAnnot is analogous to PRISMs \predref{viterbig}{2}
Use PRISM's tools for executing the direct version of the program
\end{description}

