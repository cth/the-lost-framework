% This LaTeX document was generated using the LaTeX backend of PlDoc,
% The SWI-Prolog documentation system



\section{io.pl}

\label{sec:io}

\begin{description}
    \predicate{get_data_from_file}{3}{+File, +Options, -Data}
Description: given of file composed of prolog facts,
this predicate generates a list of data given some options
By default, data predicate is in the form:
Functor(Key,LeftPosition,RightPosition,\arg{Data},...)
Type of \arg{Data} is a list

\arg{Options}: - data_position(Pos) specified in which Pos \arg{Data} is

\begin{itemize}
    \item left_position(Left) specified in which Pos Leftposition is
    \item right_position(Righ) specified in which Pos Rightposition is
    \item left_position(none) = no left position in the term
    \item right_position(none) = no right position in the term
    \item range(Min,Max): extract a range of data
    \item ranges(List_Ranges): extract a list of data given a list of Range
\end{itemize}

    \predicate{split_file}{4}{+Filename, +ChunkSize, +OutputFilePrefix, +OutputFileSuffix}
Split a file of terms into multiple files

    \predicate{split_file_fasta}{5}{+Filename, +ChunkSize, +OutputFilePrefix, +OutputFileSuffix, -ResultFiles}
Split a FASTA composed of several header ($>$ ...) into multiple files. We consider that a chunk
has been seen each time that the symbol $>$ appears at the beginning of a line
\end{description}

