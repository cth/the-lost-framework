% This LaTeX document was generated using the LaTeX backend of PlDoc,
% The SWI-Prolog documentation system



\section{genedb.pl -- working with text(prolog(ranges(_))) files}

\label{sec:genedb}

\begin{tags}
    \tag{author}
: Christian Theil Have
\end{tags}

This library contains utility predicates for working with the \verb$text(prolog(ranges(_)))$ format.
This format, which is descibed in detail in the manuals section on file formats contain
facts,

\begin{code}
gene(SequenceId,Left,Right,Strand,Frame,Extra)
\end{code}

Note that the functor 'gene' may vary.
\verb$Left$ and \verb$Right$ are integers indicating the left and right position of the gene in the genome.
They are both inclusive. \verb$Strand$, which is either '+' or '-' indicates whether the gene is located
on the direct or reverse strand. \verb$Frame$ is an integer in [1,2,3,4,5,6] which indicates the reading
frame of the gene. \verb$Extra$ is a list of key/value pairs, e.g.

\begin{code}
[product('hypothetical protein'), cid(12351)]
\end{code}

The format does not dictate which key/value pairs are required, so this may vary depending on the
application and how the database was generated.\vspace{0.7cm}

\begin{description}
    \predicate[det]{gene_sequence_id}{2}{+GeneRecord, -SequenceId}
Extract the \arg{SequenceId} field from a \arg{GeneRecord}

    \predicate[det]{gene_left}{2}{+GeneRecord, -Left}
Extract the \arg{Left} position of \arg{GeneRecord}

    \predicate[det]{gene_right}{2}{+GeneRecord, -Right}
Extract the \arg{Right} position of \arg{GeneRecord}

    \predicate[det]{gene_strand}{2}{+GeneRecord, -Strand}
Extract the \arg{Strand} (+,-) from a \arg{GeneRecord}

    \predicate[det]{gene_frame}{2}{+GeneRecord, -Frame}
Extract the \arg{Frame} from a \arg{GeneRecord}

    \predicate{gene_extra_field}{3}{+GeneRecord, +Key, -Value}
Extract the \arg{Value} of the extra field which has \arg{Key} as its functor.

    \predicate{genedb_distinct_predictions}{3}{+GeneDBFunctor, GeneEnds, PredictionsForEnd}
Find all distinct predictions in a \textit{consulted} genedb:
\arg{GeneEnds} is a list of stop codon positions for the predictions
and \arg{PredictionsForEnd} is a list of lists. The list has the same length
as \arg{GeneEnds} and each element corresponds to to an element in \arg{GeneEnds}.
The elements of the list are lists of predictions in
list predictions for the parstop codon 

    \predicate[det]{genedb_distinct_stop_codons}{2}{+GeneDBFunctor, -DistinctStops}
\arg{DistinctStops} is a list (a set) of all stop codons of predictions in the database.

    \predicate{genedb_predictions_for_stop_codon}{3}{+GeneDBFunctor, +StopMatchPattern, -Predictions}
Finds all predictions that matches \arg{StopMatchPattern}. \arg{StopMatchPattern} is a list:

\begin{code}
[+StopCodonEnd,?Strand,?Frame]
\end{code}

StopCodonEnd is position of the last nucleotide in the stop codon (i.e. right-most on direct strand and
left-most on reverse strand). Strand is '+' or '-' and frame is one of [1,2,3,4,5,6].
\end{description}

