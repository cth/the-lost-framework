% This LaTeX document was generated using the LaTeX backend of PlDoc,
% The SWI-Prolog documentation system



\section{genecode.pl -- Table of genetic codes.}

\label{sec:genecode}

This library contains various genetic code tables, i.e. what codons are start codons and stop codons.
See, \url{http://www.ncbi.nlm.nih.gov/Taxonomy/Utils/wprintgc.cgi} for further information.\vspace{0.7cm}

\begin{description}
    \predicate{genecode}{3}{+TableNum, ?Codon, ?AminoAcid}
This predicate defines the relationship between amino acids and codons with respect to a genetic code identifier, \arg{TableNum}.
\arg{Codon} is a list of of three nucleotides symbols from the alphabet [a,g,c,t].
The argument \arg{AminoAcid} is a symbol from the alphabet:

\begin{code}
[a,c,d,e,f,g,h,i,k,l,m,n,p,q,r,s,t,v,w,y,*]
\end{code}

where all symbols except \verb$*$ represents an amino acid (the name of which starts which that letter).
The special symbol \verb$*$ signifies a stop codon in the specified genetic code.

    \predicate{genecode_start_codon}{2}{+TableNum, -StartCodon}
This predicate specifies \arg{StartCodon} given a \arg{TableNum}

    \predicate{genecode_start_codons}{2}{+TableNum, -StartCodons}
\arg{StartCodons} is a list of all start codons given a genetic code \arg{TableNum}.

    \predicate{genecode_stop_codons}{2}{+TableNum, -StopCodons}
This predicate computes the list of stop codon given a \arg{TableNum}
\end{description}

