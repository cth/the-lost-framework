% This LaTeX document was generated using the LaTeX backend of PlDoc,
% The SWI-Prolog documentation system



\section{scheduler.pl -- Task scheduler}

\label{sec:scheduler}

\begin{tags}
    \tag{author}
Christian Theil Have
\end{tags}

The Task scheduler is a mechanism for running PRISM processes with dependencies in parallel.
It is used behing the scenes by the script language.
It uses the scheduler tree structure to repreent tasks and their dependencies and
calls an external shell script (\$LOST_BASE_DIR/utilities/procrun.sh) to launch new processes.
Communicates with the shell script via two files:

It writes tasks to be run to \file{pending_tasks_file.txt} and reads signals about completion
of tasks from the file \file{completed_tasks_file.txt}.\vspace{0.7cm}

\begin{description}
    \predicate{scheduler_init}{0}{}
Open communication input and output files of the scheduler and starts
the shell script based PRISM process runner.

    \predicate{scheduler_shutdown}{0}{}
Close files associated with the scheduler

    \predicate{scheduler_loop}{2}{+SchedulerTree, +RunQueue}
This is the main loop that runs the scheduler

    \predicate[private]{schedule_task}{3}{+TaskId, +Model, +Goal}
Start a new tas

    \predicate[private]{scheduler_wait}{4}{+SchedulerTree, +Queue, -UpdatedSchedulerTree, -UpdatedQueue}
This will block until a task has completed
When the task has completed, it will be removed from the queue and the scheduler tree
resulting in the updated ouput variables.

    \predicate[private]{scheduler_wait_for}{1}{-TaskId}
Read the next task id from the completed tasks file and unify \arg{TaskId} to this id
If there are no (next) completed task yet, then it will wait for 500 msecs and loop

    \predicate[private]{scheduler_wait_for_shutdown}{1}{+Count}
Gracefully wait for scheduler shutdown. Wait for hangup signal from process runner for a number of seconds
before terminating.

    \predicate[private]{scheduler_update_index}{2}{+Model, +Goal}
This will update the state and result file of the task associated \arg{Model} and \arg{Goal}
to symbolize the the task completed.
\end{description}

