% This LaTeX document was generated using the LaTeX backend of PlDoc,
% The SWI-Prolog documentation system



\section{prologdb.pl -- accessing Prolog source files.}

\label{sec:prologdb}

\begin{tags}
    \tag{author}
Christian Theil Have
\end{tags}

Predicates for working with Prolog source files, i.e. in the format \verb$text(prolog(_any))$.\vspace{0.7cm}

\begin{description}
    \predicate{file_functor}{2}{+File, ?Functor}
True if all terms in \arg{File} have the functor \arg{Functor}.

    \predicate[det]{file_functors}{2}{+File, ?Functors}
\arg{Functors} is a set of the functors in \arg{File}

    \predicate{terms_from_file}{2}{+File, -Terms}
Reads all \arg{Terms} from named file \arg{File}

    \predicate[private]{collect_stream_terms}{2}{+Stream, -Terms}
\arg{Terms} is a list of all \arg{Terms} read from \arg{Stream} untill end_of_file

    \predicate{split_prolog_file}{4}{+File, +ChunkSize, +OutputFilePrefix, +OutputFileSuffix}
Split a file of Prolog terms into multiple files, such that each file contains
a (disjunct) fraction of the Prolog terms in \arg{File}. Each of the resulting files
are valid Prolog files.

\begin{itemize}
    \item \arg{File} is the file to split up.
    \item \arg{ChunkSize} is the maximal number of terms in resulting fragment files.
    \item \arg{OutputFilePrefix} is a prefix given to resulting fragment files
    \item \arg{OutputFileSuffix} is a suffix given to resulting fragment files
\end{itemize}

    \predicate[det]{merge_prolog_files}{2}{+SeparateFiles, +MergedFile}
Merges terms from \arg{SeparateFiles} into the file \arg{MergedFile}
\end{description}

