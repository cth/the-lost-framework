%%
%% lostapis.tex
%% 
%% Made by Christian Theil Have
%% Login   <cth@noosa>
%% 
%% Started on  Fri Mar  5 11:25:51 2010 Christian Theil Have
%% Last update Fri Mar  5 11:25:51 2010 Christian Theil Have
%%

\documentclass{book}
% Creation of index
\usepackage{makeidx}
% Generation of hyperlink into a pdf file
\usepackage[colorlinks=true, urlcolor=blue,linktocpage=true]{hyperref} % Warning: Can be source of a compiled error if the \label \ref is not good 


\title{Lost API documentation}
\author{Lost Members}

\makeindex
\begin{document}
\maketitle


% Table of Contents
\addcontentsline{toc}{chapter}{Table of Contents}
\tableofcontents
%

\chapter{Setting up the lost framework}

\section{Obtaining a copy of the lost framework}

The lost framework can be obtained using \emph{git} if you have an account on the
\emph{mox} server. Assuming that you have \emph{git} installed on your
local machine, to get a copy you need to clone the central repository:
\begin{verbatim}
$ git clone ssh://your-username@mox.ruc.dk/var/git/lost.git
\end{verbatim}

\section{Configuring your copy of the lost framework}

After obtaining the lost framework, for instance by checking it out
from git, a little configuration is needed to get started.

The file \texttt{lost.pl} in the top-most directory of the copy
of the framework. In the beginning of the file there are two 
important facts you may need to change,
\begin{verbatim}
lost_config(prism_command,'prism').
lost_config(lost_base_directory, '/change/to/local/lost/dir/').
lost_config(platform, windows_or_unix).
\end{verbatim}

The option \texttt{prism\_command} should point to a the main PRISM executable
binary. If it is in your  \emph{\$PATH} then you can usually leave it unchanged.\\
\texttt{lost\_base\_directory} should be the full path of the
directory (including trailing /) containing the \texttt{lost.pl}
file.  Note, that even on windows platforms you should use forward
slash rather than backslash in the path specification. The value of
\texttt{platform} should be either \texttt{windows} or \texttt{unix}. 

To get started you can examine and run \texttt{example.pl} which 
is located in the in the \texttt{\$LOST/scripts/} directory.

\chapter{Creating lost models}

\section{Lost model conventions}\label{sec:lost_model_conventions}

Each model is located in its own subdirectory of the of the
\texttt{\{lost\}/models/} directory, henceforth called \texttt{\$MODELS}. 
So for instance, the sample model called \texttt{sample\_model1} is located in
\texttt{\{lost\}/models/sample\_model1/}. We will refer to directory 
as \texttt{\$MODEL}.

To integrate into the framework each model must provide a file called
\texttt{interface.pl}, which must be located in the same directory as
the model. \texttt{interface.pl} can then implement various predefined
predicates which serves as an entry point of using the
model. 

The supported interface predicates which a model may provide are:
\begin{itemize}
\item \texttt{lost\_best\_annotation/3}.
\item \texttt{lost\_learn/3}
\end{itemize}

By convention models are expected to store switch probabilities the
directory\texttt{\$MODEL/parameters/}. Switch parameter files should
be given the extension \texttt{.prb}.

Models are allowed to consult files with paths relative to the
\texttt{\$MODEL} directory, but should under normal circumstances
only directly consult file which are located in the 
\texttt{\$MODEL} directory or a subdirectory of it.
The exception to this is the file \texttt{\$LOST/lost.pl}. Consulting
this file gives access to all the shared APIs. 

\subsection{Model interface predicates}

This section describes predicates, that when implemented by 
the \texttt{interface.pl} provided by a model, allows the 
model provide functionalities to the  general framework.

\noindent
\paragraph{lost\_best\_annotation(+InputFileNames,+Options,+OutputFilename)}

The framework calls this predicate to obtain a ``best annotation''
from the model. The model is free to provide this annotation in
any way it sees fit. It is the models responsibility to save the
annotation to \texttt{OutputFilename}, before the completion of 
\texttt{lost\_best\_annotation}. 

\texttt{InputFileNames}: Is a list of filenames (with absolute paths),
each containing an input to the model. There is no restriction on the 
format of the files and the model is expected to be able to parse
then. Predicates to parse a wide range of fileformats are supplied 
in the \emph{io} API (see section \ref{sec:io}). 

\texttt{Options}: Is a list of facts on the form,
\texttt{option(Key,Value)}. This list is use to paramterize the model
in various ways. For convinience, option values can be checked an
extracted using the the predicates \texttt{lost\_option} and
\texttt{lost\_required\_option}, (see section \ref{sec:interface}).
Some options may be quite common and it is suggested to use the
same \texttt{Key}s for such option. An incomplete list of these common
option keys are,
\begin{itemize}
\item \texttt{parameter\_file}: Indicates that the model should use
 the switch probability associated with the \texttt{Value}. 
\end{itemize}

\texttt{OutputFilename}: Full filename which the resulting ``best
annotation'' should be saved to. The model is expected to save
the resulting annotation to this file before the completion of 
\texttt{lost\_best\_annot}. The \emph{io} API contains some
common predicates for saving annotations (see section \ref{sec:io}).

\noindent
\paragraph{lost\_learn(+InputFileNames,+Options,+OutputFilename)}

This predicates is used for training models. The model is expected to 
save the result of the training session (e.g. a switch parameter file
or similar) to \texttt{OutputFilename}. 

\texttt{InputFileNames}: Is a list of filenames (with absolute paths),
each containing an input to the model. These are used for providing 
the traning data. There is no restriction on the format of the files 
and the model is expected to be able to parse then. Predicates to
parse a wide range of fileformats are supplied in
 the \emph{io} API (see section \ref{sec:io}). 

\texttt{Options}: Is a list of facts on the form,
\texttt{option(Key,Value)}. This list is use to paramterize the model
in various ways. For convinience, option values can be checked an
extracted using the the predicates \texttt{lost\_option} and
\texttt{lost\_required\_option}, (see section \ref{sec:interface}).

\texttt{OutputFilename}: Full filename which the resulting switch
parameters or similar should be saved to. The model is expected to save
the result to this file before the completion of 
\texttt{lost\_learn}. The \emph{io} API contains some
common predicates for saving annotations (see section \ref{sec:io}).

\chapter{Lost shared APIs}

To use the lost APIs, the file \texttt{\$LOST/lost.pl} located in the
top-most \texttt{\{lost\}} directory must be consulted. Then,
APIs, which are located in the \texttt{\$LOST/shared} directory
can be consulted using the goal, 
\begin{verbatim}
lost_include_api(+APIName)
\end{verbatim}

\noindent
where \texttt{APIName} is the name of a Prolog file located
in the \texttt{\$LOST/shared/} directory except the \texttt{.pl}
extension.

\section{interface.pl}

The API provides the interface to lost models following the
conventions described in section \ref{sec:lost_model_conventions}.

get\_annotation\_file(Model, Inputs, Options, Filename)

This API provides \texttt{get\_annotation\_file/4} which is used to
retrieve the best annotation generated by a specified model with
specified parameters and input sequences. If no such file currently
exists, then the model will be run (e.g. the
\texttt{lost\_best\_annotation/3} provided by the model will be
called).

The generated annotation files are named according to a convention. 
All annotation files will be placed in the \texttt{\{lost\}/sequences/}
directory. 
The \texttt{Filename} is construed according to the following convention:
\begin{verbatim}
{Modelname}_annot_{Id}.seq
\end{verbatim}

The first time an annotation is generated the file
\texttt{annotation.idx} will be created in this directory. This file
serves as a database to map filenames of the generated annotation
files to the (models ,inputs,probability parameters) that generated
the particular annotations. This database file contains Prolog facts 
on the form,
\begin{verbatim}
fileid(Id,Filename,Model,SwitchParameters,InputFiles).
\end{verbatim}

The annotation index is automatically maintained by
\texttt{get\_annotation/4} and should normally not be edited by hand.

If annotation for a particular run of a model is not present then
\texttt{get\_annotation\_file/4} will start a new PRISM process 
that invokes the \texttt{lost\_best\_annotation} predicate provided
be the model \texttt{interface.pl} file. By the contract of model 
conventions, the model will generate the annotation and save it 
to the file indicated by the provided filename.

\section{Input-Output API}
\label{sec:io}

In this module (\texttt{io.pl}), severals predicates are defined to
manipulate \texttt{*.seq} files~:
\begin{itemize}
\item loading information from files that extracts from a file data information used as input of models (sequence annotation for example);
\item saving information into a file;
\item and maybe more. 
\end{itemize}

\subsection{Loading information from files}

\index{load\_annotation\_from\_file$\slash$ 4}
\begin{itemize}
\item \texttt{load\_annotation\_from\_file(++Type\_Info,++Options,++File,--Annotation)}:
Generate from \texttt{File} a sequence of \texttt{Annotation}. It is assumed that \texttt{File} is composed
of terms. \texttt{Type\_Info} is used to specify what format of information into file
\begin{itemize}
\item \texttt{sequence} means that information is stored into a list. For example, 
\begin{verbatim}
data(Key_Index,1,10,[a,t,c,c,c..]).
\end{verbatim}
\item \texttt{db} means that information is represented by a set of range that specified specific zone (coding region for example)
\begin{verbatim}
gb(Key_Index,1,10).
\end{verbatim}
\end{itemize}
For each \texttt{Type\_Info}, several options are available represented by the list \texttt{Options}. Options available
for \texttt{sequence}:
\begin{itemize}
\item $[]$ (default): data list is the $2^{th}$ argument of the terms and these lists of data are appended;
\item data\_position(Num) specified that data list is \texttt{Num}$^{th}$ argument of term;
\item \texttt{range(Min,Max)} extracts from the list of the complete annotation the sublist from position \texttt{Min} to
position \texttt{Max};
\item \texttt{all\_lists}: generate a list of each data list by term. Warning: \texttt{range(Min,Max)} is not support by this option.
\end{itemize}
File Example \texttt{toto.seq}:
\begin{verbatim}
data(Key_Index,1,5,[1,2,3,4,5]).
data(Key_Index,6,10,[6,7,8,9,10]).
data(Key_Index,11,15,[11,12,13,14,15]).
\end{verbatim}
Results of request are:
\begin{verbatim}
| ?- load_annotation_from_file(sequence,[data_position(4)],'toto.seq',R).
R = [1,2,3,4,5,6,7,8,9,10,11,12,13,14,15] ?
| ?- load_annotation_from_file(sequence,[data_position(4),range(4,10)],'toto.seq',R).
R = [4,5,6,7,8,9,10] ?
load_annotation_from_file(sequence,[data_position(4),all_lists],'toto.seq',R).
R = [[1,2,3,4,5],[6,7,8,9,10],[11,12,13,14,15]] ?
\end{verbatim}

Options available for \texttt{db}:
\begin{itemize}
\item $[]$ (default): first and the second element of the term defined a range. A list of 0-1 
values is generated, 0 when you are outside ranges and 1 you are inside at least one;
\item \texttt{in\_db(Letter)} replaces the default value 1 by \texttt{Letter};
\item \texttt{out\_db(Letter)} replaces the default value 0 by \texttt{Letter};
\item \texttt{range\_position(Min,Max)} allows to specify the position argument number of a term of the minimal and maximal value
of term;
\item \texttt{range(Min,Max)} extracts from the list of the complete annotation the sublist from position \texttt{Min} to
position \texttt{Max};
\end{itemize}
File Example \texttt{toto.seq}:
\begin{verbatim}
gb(3,5).
gb(7,9).
gb(8,11). % Overlap ;)
\end{verbatim}
Results of request are:
\begin{verbatim}
| ?- load_annotation_from_file(db,[],'toto.seq',R).
R = [0,0,1,1,1,0,1,1,1,1,1] ?
| ?- load_annotation_from_file(db,[in_db(c),out_db(nc)],'toto.seq',R).
R =  [nc,nc,c,c,c,nc,c,c,c,c,c]?
| ?- load_annotation_from_file(db,[range(3,7)],R).
R = [1,1,1,0,1] ?
| ?- load_annotation_from_file(db,[in_db(c),out_db(nc),range(8,16)],'toto.seq',R).
R =  [c,c,c,nc,nc,nc,nc,nc]?
\end{verbatim}


\end{itemize}

\subsection{Saving information to files}

The io API also contains some predicates for saving sequences to a
file. 

\index{save\_annotation\_to\_sequence\_file$\slash$ 4}
\begin{itemize}
\item
  \texttt{save\_annotation\_to\_sequence\_file(+KeyIndex,+ChunkSize,+Annotation,+File)}: 
Saves the data in the list given by \texttt{Annotation} to the file
\texttt{File} in \emph{sequence} format which is number of Prolog facts, on
the form:
\begin{verbatim}
data(KeyIndex,1,5, [a,t,c,c,g]).
\end{verbatim}
The first argument \texttt{KeyIndex} is used as an identifier. \texttt{ChunkSize} is the
number of elements from \texttt{Annotation} to store with each
fact. The second and third argument of a stored fact, corresponds to
the start and end position in \texttt{Annotation}. The fourth
argument of the fact is a list containing the relevant elements of the
\texttt{Annotation} list. Note that if the the length of
\texttt{Annotation} is not a multiple of \texttt{ChunkSize}, then the
last fact stored will have a shorter range.
\end{itemize}

\section{Stats API}

In this module (\texttt{stats.pl}), several predicates are defined to automatically compute
frequencies, probabilities of occurrences of nucleotides, codons ... This computation
is based on a simple counting method given a simple input data. 


\begin{itemize}
\item stats(++Data\_Type,++Options,++Data,++Input\_Counting,--Result)\index{stats$\slash 5$}
\item stats(++Data\_Type,++Options,++Data,++Input\_Counting,--Past,--Result)\index{stats$\slash 6$}
\item normalize(++Data\_Type,++List\_Counting,--Probabilities)\index{normalize$\slash 3$}
\end{itemize} 




\chapter{Models}

Pre-defined models are introduced.  These models are used to build more complex models.

\section{Parsers of Biological Data}

To extract information from different Biological database, several parsers have been designed 
to parse report of analyses available on different web-servers (Easygene, Genemark) and database (Genbank). 
These parsers generated a series of Prolog terms that can be used after that
input of different probabilistic models. \texttt{script\_parser.pl} collects different scripts to generate
different data files.

\subsection{Parser\_fna}

*.fna file of Genbank is composed of a complete genome in the FASTA format.
Parser\_fna permits to parse this *fna.file from Genbank and generate
list of terms. These terms store the genome into a Prolog list
composed of $\{a,c,g,t\}$. Two scripts are implemented:\\ 
\texttt{parser\_fna(++Name\_FNA\_File,++Name\_GBK\_File,++Options)}\index{parser\_fna$\slash$3} and \\
\texttt{parser\_fna(++Name\_FNA\_File,++Name\_GBK\_File,++Options,--OutputFile)}\index{parser\_fna$\slash$4}
Note that *.gbk file is necessary as well. This file is used to automatically extract
genome information (Genbank key and size of genome).

Output format of the generated terms are:\\
\texttt{data(Genebank\_Key,Start,End,List\_of\_Data)}.
Option \texttt{list(Number)} can be used to divide the complete
genome into several lists with a length defined by the parameter
\texttt{Number}.
Example with the E.Coli K12 genome:
\begin{verbatim}
>gi|48994873|gb|U00096.2| Escherichia coli .....
AGCTTTTCATTCTGACTGCAACGGGCAATATGTCTCTGTGTGG.....
.....
\end{verbatim}
Result by default:
\begin{verbatim}
%>gi|48994873|gb|U00096.2| Escherichia coli ....
data('U00096',1,4639675,[a,g,c,t,t,t, ...]). 
\end{verbatim}
Result when \texttt{list(280)} option is used
\begin{verbatim}
%>gi|48994873|gb|U00096.2| Escherichia .....
data('U00096',1,280,[a,g,c,t,...]).
data('U00096',281,560,[c,c,c,...]).
....
\end{verbatim}


\subsection{Parser\_ptt}

\subsection{Parser\_Easygene}

\subsection{Parser\_Genemark}

\section{Models for measurement and statistical reports}

\subsection{accuracy\_report}

The model \texttt{accuracy\_report} can be used the produce a report of
various measures of the accuracy of particular gene predictions
compared with a golden standard such a genebank. 

To use the the model, you need to call
\texttt{get\_annotation\_file/4} with the following arguments,
\begin{verbatim}
get_annotation_file(accuracy_report,
		    [ReferenceFile,PredictionFile],
		    [
		     option(reference_functor,RefFunctor),
		     option(prediction_functor,PredFunctor),
		     option(start,StartPos),
		     option(end,EndPos)
		    ],
		    OutputFile),
\end{verbatim}

\texttt{ReferenceFile} must be the full path to a file with facts
representing the ``correct predictions''.  
\texttt{PredictionFile} must be the full path to a file with facts
representing the predictions. Both files must be a a \texttt{db} type 
format, with facts on the following form,
\begin{verbatim}
functor(To, From, Strand, ReadingFrame, Name).
\end{verbatim}

Each such fact represent a prediction in the \texttt{PredictionFile} or
correct gene in the in \texttt{ReferenceFile}. 
The \texttt{functor} is a any given functor, but the
\texttt{ReferenceFile} and the \texttt{PredictionFile} should use
different functors. The \emph{To} argument represents the position in
the genome where the prediction begins (inclusive) and the
\texttt{From} argument represents the end position of the prediction
\texttt{ReadingFrame} is a integer in the range $\{1,2,3\}$
\texttt{Strand} is either \texttt{+} for the forward strand or
    \texttt{-} for the reverse strand. 

\texttt{get\_annotation\_file} for the \texttt{accuracy\_report} model
must be called with four mandatory options:
\begin{itemize}
\item \texttt{reference\_functor}: The functor used in the \texttt{ReferenceFile}
\item \texttt{prediction\_functor}: The functor used in the
  \texttt{PredictionFile}
\item \texttt{start}: An integer corresponding to the beginning of the range on which accuracy
  should be measured.
\item \texttt{end}: An integer corresponding to the end of the range
  (inclusive) on which accuracy should be measured.
\end{itemize}


\chapter{Notes and stuff}

\section{Feature wish list}

\begin{itemize}
\item A \texttt{data} directory instead of a \texttt{sequences}
  directory.
\item Division of models into models (probabilistic models) and nodes
  (which are just data processing).
\item kind of \texttt{make.pl} file that manages the consulting of all the
useful file (lost, /shared/*.*, /scripts/*.*). Just let to the user
to do consulting of model file. If it is possible, have a very simple
interaction with the user to set the lost location, set platform, or list
models or data available.
\end{itemize}



\printindex

\end{document}
