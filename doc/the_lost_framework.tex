%%
%% lostapis.tex
%% 
%% Made by Christian Theil Have
%% Login   <cth@noosa>
%% 
%% Started on  Fri Mar  5 11:25:51 2010 Christian Theil Have
%% Last update Fri Mar  5 11:25:51 2010 Christian Theil Have
%%

\documentclass{article}
\title{Lost API documentation}


\begin{document}
\maketitle

\section{Setting up the lost framework}

After obtaining the lost framework, for instance by checking it out
from git, a little configuration is needed to get started.

The file \texttt{lost.pl} in the top-most directory of the copy
of the framework. In the beginning of the file there are two 
important facts you may need to change,
\begin{verbatim}
lost_config(prism_command,'/opt/prism/bin/prism').
lost_config(lost_base_directory, '/home/cth/code/lost/').
\end{verbatim}

The option \texttt{prism\_command} should point to a the main PRISM executable
binary. \\
\texttt{lost\_base\_directory} should be the full path of the
directory (including trailing /) containing the \texttt{lost.pl}
file. 

To get started you can examine and run \texttt{example.pl} which 
is also located in the top-most directory.

\subsection{Lost model conventions}
\label{sec:lost_model_conventions}

Each model is located in its own subdirectory of the of the
\texttt{\{lost\}/models/} directory. So for instance, the sample model
called \texttt{sample\_model1} is located in
\texttt{\{lost\}/models/sample\_model1/}.

To integrate into the framework each model must provide a file called
\texttt{interface.pl}, which must be located in the same directory as
the model. \texttt{interface.pl} can then implement various predefined
predicates which serves as an entry poinit of using the
model. Currently, the only supported predicate is
\texttt{lost\_best\_annotation/3}.

In addition models are expected to store switch probabilities the
subdirectory \texttt{parameters/} of the model directory. Switch
parameter files should be given the extension \texttt{.prb}.



\subsubsection{Model interface predicates}

This section describes predicates, that when implemented by 
the \texttt{interface.pl} provided by a model, allows the 
model provide functionalities to the  general framework.

\noindent
\paragraph{lost\_best\_annotation(+ParameterFile,+InputFiles,+OutputFilename)}

The framework calls this predicate to obtain a ``best annotation''
from the model. The model is free to provide this annotation in
any way it sees fit. It is the models responsibility to save the
annotation to \texttt{OutputFilename}.

\texttt{OutputFilename}: Full filename which the resulting ``best
  annotation'' should be saved to.

\texttt{ParameterFile}: Is the full name of a file containing the
parameters the model should use. The model may disregard this argument
if the it does not deal with parameters. 

\texttt{InputFiles}: Is a list filename, each containing inputs to the
model.

\paragraph{Note: I think it might be a good idea to add an ``extra
  options'' parameter to the lost\_best\_annotation, to enable a model
  to be parameterized.}

\section{lost shared APIs}

To use the lost APIs, the file \texttt{lost.pl} located in the
top-most \texttt{\{lost\}} directory must be consulted. Then,
APIs, which are located in the \texttt{\{lost\}/shared} directory
can be consulted using the goal, 
\begin{verbatim}
lost_include_api(APIName).
\end{verbatim}

\noindent
where \texttt{APIName} is the name of a Prolog file located
in the \texttt{\{lost\}/shared/} directory except the \texttt{.pl}
extension.

\subsection{interface.pl}

The API provides the interface to lost models following the
conventions described in section \ref{sec:lost_model_conventions}.

\subsubsection{get\_annotation\_file/4}

This API provides \texttt{get\_annotation\_file/4} which is used to
retrieve the best annotation generated by a specified model with
specified parameters and input sequences. If no such file currently
exists, then the model will be run (e.g. the
\texttt{lost\_best\_annotation/3} provided by the model will be
called).


The generated annotation files are named according to a convention. 
All annotation files will be placed in the \texttt{\{lost\}/sequences/}
directory. 
The \texttt{Filename} is construed according to the following convention:
\begin{verbatim}
{Modelname}_annot_{Id}.seq
\end{verbatim}

The first time an annotation is generated the file
\texttt{annotation.idx} will be created in this directory. This file
serves as a database to map filenames of the generated annotation
files to the (models ,inputs,probability parameters) that generated
the particular annotations. This database file contains Prolog facts 
on the form,
\begin{verbatim}
fileid(Id,Filename,Model,SwitchParameters,InputFiles).
\end{verbatim}

The annotation index is automatically maintained by
\texttt{get\_annotation/4} and should normally not be edited by hand.

If annotation for a particular run of a model is not present then
\texttt{get\_annotation\_file/4} will start a new PRISM process 
that invokes the \texttt{lost\_best\_annotation} predicate provided
be the model \texttt{interface.pl} file. By the contract of model 
conventions, the model will generate the annotation and save it 
to the file indicated by the provided filename.

\subsection{The sequence API}

Not implemented/included yet.

\subsection{The accuracy API}

Not implemented/included yet.

\end{document}
