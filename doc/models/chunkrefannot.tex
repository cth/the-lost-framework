% This LaTeX document was generated using the LaTeX backend of PlDoc,
% The SWI-Prolog documentation system



\subsection{chunk_ref_annot}

\label{sec:chunkrefannot}

\begin{tags}
    \tag{author}
: Christian Theil Have
\end{tags}

Augments predicted genes or orfs with information from a golden standard/reference file.

Currently, there are two principal ways of using this module. For both methods, it
is the case that their InputFiles is a list

\begin{code}
InputFiles = [ ReferenceFile, PutativeFile ]
\end{code}

The two methods can be summarized as follows:

\begin{enumerate}
    \item \verb$add_reference_track$ : adding an "extra" field with a reference annotation to a each "gene"/orf in a file.
E.g. if we have a \verb$PutativeFile$ with an entry,

\begin{code}
gene(seqid, 1, 10, +, 1, []).
\end{code}

And a golden standard \verb$ReferenceFile$ with the entry,

\begin{code}
ref(seqid,4,10,+,1,[]).
\end{code}

Then the updated gene record (OutputFile) will look as follows,

\begin{code}
gene(seqid,1,10,+,1,[ref_annot([0,0,0,0,1,1,1,1,1,1])]).
\end{code}

A term with a list is added to the \verb$extra$ list (sixth argument).
The list has zeroes in all non-coding positions and ones in the coding positions (of the same strand reading frame).
    \item \verb$maching_genes$ : Write to the OutputFile all genes from the \verb$ReferenceFile$ which partially overlap with a gene from the \verb$PutativeFile$ (must have same Strand and Frame).
This may be useful for instance when creating cross-validation sets, e.g. if you want to measure accuracy wrt to a smaller set.
\end{enumerate}

\vspace{0.7cm}

\begin{description}
    \predicate{add_reference_track}{3}{+InputFiles, +Options, +OutputFile}
Type signature:

\begin{code}
InputFiles:
    1. text(prolog(ranges(gene)))
    2. text(prolog(ranges(gene)))
OutputFile:
    text(prolog(ranges(gene)))
Options:
\end{code}

\begin{code}
InputFiles = [ PutativeFile, ReferenceFile ]
\end{code}

adds and extra field to each gene in \verb$PutativeFile$ if the are a (partically) overlapping gene in \verb$ReferenceFile$ with same strand+frame.
The extra field contains a list, which
has zeroes in all non-coding positions and ones in the coding positions (of the same strand reading frame).

    \predicate{matching_genes}{3}{+InputFiles, +Options, +OutputFile}
Type signature:

\begin{code}
InputFiles:
    1. text(prolog(ranges(gene)))
    2. text(prolog(ranges(gene)))
OutputFile:
    text(prolog(ranges(gene)))
Options:
\end{code}

\arg{InputFiles} = [ File1, File2 ]
\arg{OutputFile} is all the gene entries in File1 which have identical the Left, Right, Frame and Strand to an of entry in File2.
\end{description}

