% This LaTeX document was generated using the LaTeX backend of PlDoc,
% The SWI-Prolog documentation system



\subsection{blastgf}

\label{sec:blastgf}

\begin{tags}
\mtag{author}- : Christian Theil Have \\- : Ole Torp Lassen
\end{tags}

Blast based gene finder.

This gene finder predicts genes based on blast hits.

This is based on Oles (now defunct it seeems) chunk_aa_conservtion model, but considerably rewritten by Christian Theil Have.
The \verb$blastgf$ model only contains the probabilistic HMM and prism model related to prediction of
genes based on blast matches and more mundane tasks like preprocessing, blast xml parsing, running of
blast have been omitted from this model.

FIXME: More documentation needs to be done.\vspace{0.7cm}

\begin{description}
    \predicate{annotate_single_track}{3}{+InputFiles, +Options, +OutputFile}
Type signature:

\begin{code}
InputFiles:
    1. text(prolog(prism_parameters))
    2. text(prolog(ranges(gene)))
OutputFile:
    text(prolog(ranges(gene)))
Options:
\end{code}

\arg{InputFiles} = [ ParamsFile, OrfsFile ]
This task creates annotations for the ORFs of OrfsFile.
The Orfs in OrfsFile are expected to have an extra field, \verb$identity_seq$, a list of 0 and 1's in which ones identity positions with a blast hit.
Such an extra field can be added with the \verb$orf_blaster$ model.
If part of and ORF in OrfsFile is predicted as coding, then the ORF will be written to \arg{OutputFile}.
The term written to output file will have an additional extra field \verb$blastgf(AnnotList)$ where \verb$AnnotList$ is a list of 0 and 1, where one 0 means predicted as non-coding and 1 means predicted as coding.

    \predicate{annotate_multi_track}{3}{+InputFiles, +Options, +OutputFile}
Type signature:

\begin{code}
InputFiles:
    1. text(prolog(prism_parameters))
    2. text(prolog(ranges(gene)))
OutputFile:
    text(prolog(ranges(gene)))
Options:
\end{code}

\arg{InputFiles} = [ ParamsFile, OrfsFile ]
This task creates annotations for the ORFs of OrfsFile.
The Orfs in OrfsFile are expected to have an extra fields, \verb$identity_seq$, a list of integers 0-8, there the number indicates the number identity positions with a blast hits to one of eight other organisms.
In the future, the number of organisms may be made configurable.
Such an extra field can be added with the \verb$orf_blaster$ model.
If part of and ORF in OrfsFile is predicted as coding, then the ORF will be written to \arg{OutputFile}.
The term written to output file will have an additional extra field \verb$blastgf(AnnotList)$ where \verb$AnnotList$ is a list of 0 and 1, where one 0 means predicted as non-coding and 1 means predicted as coding.

    \predicate{parallel_annotate_multi_track}{3}{+InputFiles, +Options, +OutputFile}
Type signature:

\begin{code}
InputFiles:
    1. text(prolog(prism_parameters))
    2. text(prolog(ranges(gene)))
OutputFile:
    text(prolog(ranges(gene)))
Options:
\end{code}

Same as \verb$annotate_multi_track$ but running with 10 parallel threads.
This task is deprecated.

    \predicate{parallel_annotate_single_track}{3}{+InputFiles, +Options, +OutputFile}
Type signature:

\begin{code}
InputFiles:
    1. text(prolog(prism_parameters))
    2. text(prolog(ranges(gene)))
OutputFile:
    text(prolog(ranges(gene)))
Options:
\end{code}

Same as \verb$annotate_single_track$ but running with 10 parallel threads.
This task is deprecated.

    \predicate{learn_single_track}{3}{+InputFiles, +Options, +OutputFile}
Type signature:

\begin{code}
InputFiles:
    1. text(prolog(ranges(gene)))
OutputFile:
    text(prolog(prism_parameters))
Options:
\end{code}

The Orfs in the input file are expected to have an extra field, \verb$identity_seq$, a list of 0 and 1's in which ones identity positions with a blast hit.
Also, the Orfs in the input file are expected to have an extra field, \verb$ref_annot$, a list of 0 and 1's in which ones indicate a that (part of) the orf is real gene.
The result of running the task is PRISM parameter file.

    \predicate{learn_multi_track}{3}{+InputFiles, +Options, +OutputFile}
Type signature:

\begin{code}
InputFiles:
    1. text(prolog(ranges(gene)))
OutputFile:
    text(prolog(prism_parameters))
Options:
\end{code}

The Orfs in the input file are expected to have an extra field, \verb$identity_seq$, a list of integers 0-8, there the number indicates the number identity positions with a blast hits to one of eight other organisms.
Also, the Orfs in the input file are expected to have an extra field, \verb$ref_annot$, a list of 0 and 1's in which ones indicate a that (part of) the orf is real gene.
The result of running the task is PRISM parameter file.

    \predicate{parallel_learn_single_track}{3}{+InputFiles, +Options, +OutputFile}
Type signature:

\begin{code}
InputFiles:
    1. text(prolog(ranges(gene)))
OutputFile:
    text(prolog(prism_parameters))
Options:
\end{code}

Same as \verb$learn_single_track$, but running with 10 parallel threads.

    \predicate{parallel_learn_multi_track}{3}{+InputFiles, +Options, +OutputFile}
Type signature:

\begin{code}
InputFiles:
    1. text(prolog(ranges(gene)))
OutputFile:
    text(prolog(prism_parameters))
Options:
\end{code}

Same as \verb$learn_multi_track$, but running with 10 parallel threads.
\end{description}

