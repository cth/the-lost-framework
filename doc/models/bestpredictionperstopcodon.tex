% This LaTeX document was generated using the LaTeX backend of PlDoc,
% The SWI-Prolog documentation system



\subsection{best_prediction_per_stop_codon}

\label{sec:bestpredictionperstopcodon}

\begin{tags}
    \tag{author}
: Christian Theil Have
\end{tags}

This is a simple model for excluding several overlapping predictions
with the same stop-codon. Instead, given a set of predictions with
scores (as extra field) the model selects the best scoring gene prediction
for each stop codon.
I.e. the final set of predictions will only contain one prediction per
stop codon.

The functor of the extra field that contains the score of the prediction, needs
to be specified using an option.\vspace{0.7cm}

\begin{description}
    \predicate{filter}{3}{+InputFiles, +Options, +OutputFile}
Type signature:

\begin{code}
InputFiles:
    1. text(prolog(ranges(gene)))
OutputFile:
    text(prolog(ranges(gene)))
Options:
    - prediction_functor (default value: auto)
    - score_functor (default value: not_set)
\end{code}

For each distinct stop codon in the predictions in the the input file, write prediction which has has the highest
score to the output file.
The functor of predictions in the input file will be automatically inferred if option \verb$prediction_functor=auto$ and the file contains only the same type of functor. Otherwise the value of \verb$prediction_functor$ should be set.
The \verb$score_functor$ option \textit{must} be set. The predictions are expected to have an extra field indicating the score of a prediction as numeric value, .e.g. in the example below, the you would use \verb$score_functor(score)$.

\begin{code}
prediction(org,123,456,+,1,[score(0.99)])
\end{code}

\end{description}

