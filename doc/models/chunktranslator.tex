% This LaTeX document was generated using the LaTeX backend of PlDoc,
% The SWI-Prolog documentation system



\subsection{chunk_translator}

\label{sec:chunktranslator}

\begin{tags}
    \tag{author}
: Ole Torp Lassen
\end{tags}

Translates a file of frame specific nucleotide chunks into a file of equivalent fastaformatted amino-acid sequences.\vspace{0.7cm}

\begin{description}
    \predicate{translate}{3}{+InputFiles, +Options, +OutputFile}
Type signature:

\begin{code}
InputFiles:
    1. text(prolog(ranges(gene)))
OutputFile:
    text(fasta(ffa))
Options:
    - mode (default value: 0)
    - genecode (default value: 11)
\end{code}

\begin{code}
 InputFiles = [ ChunkFile ]
\end{code}

Translates each chunk in ChunkFile to get the amino acid sequence.
The option \verb$mode$ determines whether to run translate entire chunk \verb$mode=0$ or the longest orf in chunk \verb$mode=1$.
The translated chunks are written \arg{OutputFile} which in a multi-fasta format.
\end{description}

